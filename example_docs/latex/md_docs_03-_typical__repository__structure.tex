{\itshape {\bfseries{R\+E\+Q\+U\+I\+R\+ED\+:}}} {\ttfamily packages.\+json} -\/ houses package names and their checksums

package.\+json 
\begin{DoxyCode}{0}
\DoxyCodeLine{\{}
\DoxyCodeLine{    "{}uspm"{}:"{}9c19a317129b8c42ff9358f809beef01"{}, }
\DoxyCodeLine{    "{}cjson"{}:"{}97771ff98b30d382bc1917a8089d2e11"{}}
\DoxyCodeLine{\}}
\end{DoxyCode}


The rest are basically just .uspm packages that you\textquotesingle{}d want on the server. 
\begin{DoxyCode}{0}
\DoxyCodeLine{.}
\DoxyCodeLine{+-\/-\/ cjson.uspm}
\DoxyCodeLine{+-\/-\/ libcurl.uspm}
\DoxyCodeLine{+-\/-\/ openssl.uspm}
\DoxyCodeLine{+-\/-\/ packages.json}
\DoxyCodeLine{+-\/-\/ tar-\/tbc.uspm}
\DoxyCodeLine{+-\/-\/ uspm.uspm}
\DoxyCodeLine{+-\/-\/ uspp-\/src.uspm}
\end{DoxyCode}


If you\textquotesingle{}d like you can keep archived versions of packages in the {\ttfamily archive} folder. However, the U\+S\+PM standard does not require that such a folder exist. 